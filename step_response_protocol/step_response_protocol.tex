\documentclass{article}
\usepackage[margin=1in]{geometry}
\usepackage[shortlabels]{enumitem}

\begin{document}

\begin{center}
{\bf \Large Beluga Step Response Protocol} \\ \smallskip
5 February 2017 \\
\rule{\textwidth}{1pt}
\end{center}

\section*{Purpose and Overview}
The purpose of the step response test on the Beluga robot is to verify and tune the dynamic model for the robot.
Data collected from the robot will be compared to the output of a simulation using the dynamic model to determine its accuracy in aid in improving the model.

The tests are conducted by submerging the robot and supplying all four motors with an identical step input.
Tests are run at 25\%, 50\%, 75\%, and 100\% of the robot's maximum thrust (determined by the safe limits of the power supply and estimated power draw).
The tests are done in a controlled environment (swimming pool) with team members in the water to catch the robot in the case of a problem.

\section*{Physical Setup}
Start the robot at one end of the pool, above the water, with another teammate at the other end of the pool.
Once the robot receives the step test command, the motors will start spinning in 5 seconds.
During this time, submerge the robot by pushing it down gently below the surface.
Allow the robot to complete its step response and then bring it to the surface so that the scripts can be stopped and the data saved.

\section*{Code Operation}
A number of scripts need to be run in order to conduct the tests and record data.
It is recommended to use a \texttt{screen} session to manage all of the scripts under a single \texttt{ssh} connection.
The script setup and execution works as follows:

\begin{enumerate}
\item{ \texttt{ssh} into the Beluga. 
On the lab machine, this is aliased as the command \texttt{orca}.
Ensure that the router is plugged in and the computer is connected to the wifi named "the-ocean"
}
\item{
Start a screen session using the command \texttt{screen}.
Control-a c opens a new screen tab and control-a [WINDOW\_NUM] switches windows.
}
\item{
Start up the sensors in separate tabs with the following commands: \vspace{-15pt}
\begin{center}
\begin{tabular}{lll}
IMU: & \texttt{roslaunch beluga\_ros vn100.launch port:=[PORT]} & PORT is most likely /dev/ttyUSB1 \\
GPS: & \texttt{roslaunch beluga\_ros gps.launch port:=[PORT]} & PORT is most likely /dev/ttyUSB0 \\
\end{tabular}
\end{center}
}
\item{
For each test, run (in this order and quickly one after the other):
\begin{enumerate}[i.]
\item \texttt{rosbag record -a -o [TEST\_NAME]}
\item \texttt{roslaunch beluga\_ros ekf2.launch}
\item \texttt{roslaunch beluga\_nav step\_test.launch top:=[POWER] bottom:=[POWER] right:=[POWER] \\ left:=[POWER] dur:=[STEP\_TIME]}
\end{enumerate}
All of these commands should be killed (with control-c) once the robot surfaces and be re-run for each test.
The rosbag record should be run in the folder that will end up containing the data. \\
Note: currently POWER should be input backward (between 0 and -1), think this is a wiring issue.
}
\end{enumerate}

\section*{Notes}
For a 25m pool, the following step times should work (i.e. not hit the wall):
\begin{center}
\begin{tabular}{ll}
Power & Time \\ \hline
25\% & 15s \\
50\% & 12s \\
75\% & 12s \\
100\% & 10s \\
\end{tabular}
\end{center}


\end{document}
